\documentclass{article}
\usepackage[utf8]{inputenc}
\usepackage[russian]{babel}
\usepackage{graphicx}
\usepackage{listings}
\usepackage{xcolor}
\usepackage{hyperref}
\usepackage{geometry}

\geometry{a4paper, left=25mm, right=15mm, top=20mm, bottom=20mm}

\definecolor{codegreen}{rgb}{0,0.6,0}
\definecolor{codegray}{rgb}{0.5,0.5,0.5}
\definecolor{codepurple}{rgb}{0.58,0,0.82}

\lstdefinestyle{mystyle}{
    backgroundcolor=\color{white},   
    commentstyle=\color{codegreen},
    keywordstyle=\color{magenta},
    numberstyle=\tiny\color{codegray},
    stringstyle=\color{codepurple},
    basicstyle=\ttfamily\footnotesize,
    breakatwhitespace=false,         
    breaklines=true,                 
    captionpos=b,                    
    keepspaces=true,                 
    numbers=left,                    
    numbersep=5pt,                  
    showspaces=false,                
    showstringspaces=false,
    showtabs=false,                  
    tabsize=2,
    frame=single
}

\lstset{style=mystyle}

\title{Документация к проекту "Алгоритм Левита"}
\author{}
\date{}

\begin{document}

\maketitle

\section{Введение}
Проект реализует алгоритм Левита для поиска кратчайших путей в графе с возможностью работы в клиент-серверной архитектуре. Основные компоненты:

\begin{itemize}
    \item Серверная часть (принимает граф, вычисляет кратчайшие пути)
    \item Клиентская часть (генерирует тестовые графы, отправляет на сервер)
    \item Общие методы (реализация алгоритма Левита)
\end{itemize}

\section{Компиляция и запуск}

\subsection{Требования}
\begin{itemize}
    \item Компилятор C++17 (g++ или clang)
    \item CMake 3.10+
    \item Библиотека nlohmann/json
\end{itemize}

\subsection{Сборка проекта}
\begin{lstlisting}[language=bash]
cd build
cmake ..
make
\end{lstlisting}

После сборки в папке \texttt{build} появятся исполняемые файлы:
\begin{itemize}
    \item \texttt{Server} - серверная часть
    \item \texttt{Client} - клиентская часть
\end{itemize}


\section{Тестирование}
Клиентская часть включает 4 теста:
\begin{enumerate}
    \item Простой связный граф
    \item Граф с отрицательными весами
    \item Большой случайный граф (100 вершин)
    \item Граф с недостижимыми вершинами
\end{enumerate}

Результаты тестов сохраняются в файлы:
\begin{itemize}
    \item \texttt{result\_simple.txt}
    \item \texttt{result\_negative.txt}
    \item \texttt{result\_large\_random.txt}
    \item \texttt{result\_unreachable.txt}
\end{itemize}

\section{Заключение}
Проект предоставляет эффективную реализацию алгоритма Левита с возможностью удаленного выполнения вычислений. Клиент-серверная архитектура позволяет легко интегрировать решение в существующие системы.

\end{document}